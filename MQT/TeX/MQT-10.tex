\documentclass{article}
\usepackage{graphicx}
\usepackage{amsmath, amssymb}
\usepackage{mathtools}
\usepackage{derivative}
\usepackage{enumitem}
\usepackage{amsfonts}
\usepackage{dsfont}

\newcommand{\R}{\mathbb{R}}
\newcommand{\C}{\mathbb{C}}
\newcommand{\F}{\mathcal{F}}
\newcommand{\ex}{\textit}
\newcommand{\sol}{\\ \textbf{Solution: }}
\newcommand{\proof}{\\ \textbf{Proof: }}
\newcommand*{\eqdef}{\ensuremath{\overset{\mathclap{\text{\tiny def}}}{=}}}
\usepackage{graphicx} % Required for inserting images
\newcommand{\ssubset}{\subset\joinrel\subset}
\newcommand{\ppdv}[2]{\frac{\partial^2 #1}{\partial{#2}^2}}

\title{Exercises - Mathematical Quantum Theory}
\author{Simone Coli - 6771371}
\date{Sheet 10}
\begin{document}
\maketitle

\section{Exercise}
(a) Given a PVM ${\{ P(\Omega) \}}_{\Omega \in \mathcal{B} (\R)}$ let $\Phi: S(\R) \to \mathcal{L} (\mathcal{H})$ be the associated measurable functional calculus. Which by definition have the form:
    \[
        \Phi(f) = \int_\R f(\lambda) dp(\lambda) := \sum_i c_i P(\Omega_i)
    \]
where in this case $f$ is a simple function. We want to prove that $\Phi$ is a $C^*-$algebra homeomorphism, i.e.:
\begin{itemize}
    \item $\Phi$ is linear: from the linearity of integrals we have that it is linear.
    \item $\Phi(1) = \mathds{1}_\mathcal{H}$:
    \[
        \Phi(\mathbf 1) = \int_\R 1 dp (\lambda) = \int_\R dp (\lambda) = P(\R) = \mathds{1}_\mathcal{H}
    \]
    \item For all $f, g \in S(\R)$, we have $\Phi(fg) = \Phi(f)\Phi(g)$: from the definition of a simple function we can express $f = \sum_j \alpha_j \chi_{\Omega_j}$ and $g = \sum_k \beta_k \chi_{\Gamma_k}$ which allows us to consider
    \[
        \begin{split}
            \Phi(fg) &= \Phi \left(\sum_j \alpha_j \chi_{\Omega_j} \sum_k \beta_k \chi_{\Gamma_k}\right) = \Phi \left(\sum_{j,k} \alpha_j \beta_k \chi_{\Omega_j} \chi_{\Gamma_k}\right) =\\
            & = \Phi \left(\sum_{j,k} \alpha_j \beta_k \chi_{\Omega_j \cap \Gamma_k}\right)
        \end{split}
    \]
    using now the definition of functional calculus for simple functions:
    \[
        \Phi(fg) = \sum_{j,k} \alpha_j \beta_k P(\Omega_j \cap \Gamma_k) = \sum_{j,k} \alpha_j \beta_k P(\Omega_j) P (\Gamma_k) = \Phi(f)\Phi(g)
    \]
    \item For all $f \in S(\R), \Phi(\bar f) = \Phi(f)^*$: again using the definition a simple function we have
    \[
        f = \sum_j \alpha_j \chi_{\Omega_j}, \quad \bar f = \overline{\sum_j \alpha_j \chi_{\Omega_j}} = \sum_j \overline{\alpha_j} \chi_{\Omega_j}
    \]
    from the properties of complex conjugate and the fact that the characteristic function real valued. Then, by the definition of functional calculus for simple functions:
    \[
        \Phi(\bar f) = \Phi \left( \sum_j \overline{\alpha_j} \chi_{\Omega_j} \right) = \sum_j \overline{\alpha_j} P(\Omega_j)  = \sum_j \overline{\alpha_j} P(\Omega_j)^* = \Phi(f)^*
    \]
\end{itemize} 
(b) Now let $f, g  : \R \to \C$ be measurable, but not necessarily bounded. We want to show that the $\Phi(f) \Phi(g) = \Phi (g)\Phi(f)$ on $\mathcal{D}_f \cap \mathcal{D}_g$:

Let us consider two sequences $f_k \subset \mathcal{D}_f, g_k \subset \mathcal{D}_g$ converging respectively to $f \in \mathcal{D}_f, g \in \mathcal{D}_g$ and such that $f_k = f \chi_{\Omega_k}, g_k = g \chi_{\Gamma_k}$. By the definition of $(\Phi(\cdot), \mathcal{D})$ we set $\Phi(f) = \lim_{k \to \infty} \Phi(f_k)$, then:
\[
    \begin{split}
        \Phi(f) \Phi(g) &= \lim_{k \to \infty} \Phi(f_k) \Phi(g_k) = \lim_{k \to \infty} \Phi(f\chi_{\Omega_k}) \Phi(g \chi_{\Gamma_k}) =\\
        & = \lim_{k \to \infty} \Phi(f\chi_{\Omega_k}\, g\, \chi_{\Gamma_k}) = \lim_{k \to \infty}  \Phi(g \chi_{\Gamma_k}) =  \Phi(f\chi_{\Omega_k}) =\\
        &= \lim_{k \to \infty}  \Phi(g_k) \Phi(f_k) = \Phi(f) \Phi(g)
    \end{split}
\]
since from their definition, $f_k, g_k$ are bounded.
\section{Exercise}

Let us take $\mathcal{H} = L^2([0,1])$ and consider the Volterra operator
\[
    Vf(x) = \int_0^1 f(y) dy.
\]
(a) we want to show that such an operator is bounded on $\mathcal{H}$.

The operator norm
\[
    \begin{split}
        \| V \|^2 &= \sup_{{\| f \|}_{L^2} = 1} \|Vf \|_{L^2}^2 =\\
        &= \sup_{{\| f \|}_{L^2} = 1} \int_{0}^{1} \left| \int_{0}^{x} f(y) dy \right|^2 dx \leq  \sup_{{\| f \|}_{L^2} = 1} \int_{0}^{1} {\left( \int_{0}^{x} |f(y)| dy \right)}^2 dx
    \end{split}
\]
by Cauchy-Schwartz:
\[
    \int_{0}^{1} {\left( \int_{0}^{x} |f(y)| dy \right)}^2 dx \leq \int_{0}^{1} {\|f(x)\|}_{L^2} x\, dx < \infty
\]
(b)
\section{Exercise}
\section{Exercise}
\end{document}