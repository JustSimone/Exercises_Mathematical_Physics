\documentclass{article}
\usepackage{graphicx} % Required for inserting images
\usepackage{amsmath}
\usepackage{mathtools}
\usepackage{amsfonts}
\usepackage{derivative}

\title{Exercise sheet Mathematical Quantum Theory - 1}
\author{Simone Coli}
\date{October $27^{\mbox{th}}$, 2023}

\begin{document}

\maketitle
\section{Solution:}
\subsection*{a)}
Let us assume that a general solution to the Schrödinger equation
\begin{equation}\label{eq:1}
    i \partial_t \psi_t = \hat H \psi_t
\end{equation}
has the form $\psi_t = \psi_0 e^{-iEt}$. By inserting this inside \eqref{eq:1} we get that $E = -B\sigma_x$. The diagonalization of $\sigma_x$ gives us the two eigenvalues $\lambda_{\pm} = \pm 1$ and the two orthonormal eigenvectors
\[
    \mathbf v_1 = \frac{1}{\sqrt{2}}
    \begin{pmatrix}
        1\\
        1
    \end{pmatrix}
    ,\quad
    \mathbf v_2 = \frac{1}{\sqrt{2}}
    \begin{pmatrix}
        1\\
        -1
    \end{pmatrix}.
\]
So that the general solution takes the form:
\begin{equation}\label{eq:2}
    \psi_t = \frac{1}{\sqrt{2}} 
    \begin{pmatrix}
        1\\
        1
    \end{pmatrix}
    \psi_+ e^{iBt} + \frac{1}{\sqrt{2}} 
    \begin{pmatrix}
        1\\
        -1
    \end{pmatrix}
    \psi_- e^{-iBt}
\end{equation}
If we now impose the initial condition on \eqref{eq:2} we get the system
\[
    \begin{cases}
      \,\frac{\psi_+ +\, \psi_-}{\sqrt{2}} = 1\\
      \,\frac{\psi_+ -\, \psi_-}{\sqrt{2}} = 0
    \end{cases}\,.
\]
From which we get $\psi_+ = \psi_- = 1/\sqrt{2}$, and \eqref{eq:2} becomes
\begin{equation}\label{eq:3}
    \psi_t = \frac{1}{2} 
    \begin{pmatrix}
        1\\
        1
    \end{pmatrix}
    e^{iBt} + \frac{1}{2} 
    \begin{pmatrix}
        1\\
        -1
    \end{pmatrix}
    e^{-iBt} = 
    \begin{pmatrix}
        \cos{(Bt)}\\
        i \sin{(Bt)}
    \end{pmatrix}
\end{equation}
\subsection*{b)}
The probability of having as outcome of our measurement the eigenvalue $+1$ or $-1$ is given by the projection of the solution on the eigenvector (related to those eigenvalue) of the operator $\sigma_z$. That is:
\[
    P(+1)_{\sigma_z} = |\langle 1| \psi_t\rangle|^2 =
    \begin{vmatrix}
    \begin{bmatrix}
        \begin{pmatrix}
            1 & 0
        \end{pmatrix}
        \begin{pmatrix}
            \cos{(Bt)}\\
            i\sin{(Bt)}
        \end{pmatrix}
    \end{bmatrix}
    \end{vmatrix}^2 
    = \cos^2{(Bt)}
\]
\[
    P(-1)_{\sigma_z} = |\langle -1| \psi_t\rangle|^2 =
    \begin{vmatrix}
    \begin{bmatrix}
        \begin{pmatrix}
            0 & -1
        \end{pmatrix}
        \begin{pmatrix}
            \cos{(Bt)}\\
            i\sin{(Bt)}
        \end{pmatrix}
    \end{bmatrix}
    \end{vmatrix}^2 
    = \sin^2{(Bt)}
\]
\subsection*{c)}
The expectation value of $\sigma_z$ is just given by the intern product of the solution with the solution projected on 
\[
    \begin{split}
        \langle \sigma_z \rangle_{\psi_t} &= \langle \psi_t | \sigma_z | \psi_t \rangle =\\
        &= \begin{pmatrix}
            \cos{(Bt)} & -i \sin{(Bt)}
        \end{pmatrix}
        \begin{pmatrix}
            1 & 0\\
            0 & -1
        \end{pmatrix}
        \begin{pmatrix}
            \cos{(Bt)}\\ 
            i \sin{(Bt)}
        \end{pmatrix}\\
        &= \cos^2{(Bt)} - \sin^2{(Bt)}
    \end{split}
\]
\subsection*{d)}
The value of the variance of for the two operator is given by:
\[
    \begin{split}
        (\Delta \sigma_x)_{\psi_t} &= \langle \psi_t, (\sigma_x - \langle \psi_t, \sigma_x \psi_t \rangle)^2 \psi_t\rangle\\
        &= \langle \psi_t, \sigma_x^2 \psi_t \rangle - 4\langle \psi_t, \cos^2{Bt} \sin^2{Bt} \, \psi_t \rangle - 4i\langle \psi_t, \sigma_x \cos{Bt} \sin{Bt}\psi_t \rangle.
    \end{split}
\]
Given the value of the expected value of $\sigma_x$ to be
\[ 
    \begin{split}
        \langle \sigma_x \rangle_{\psi_t} &=
    \begin{pmatrix}
            \cos{(Bt)} & -i \sin{(Bt)}
        \end{pmatrix}
        \begin{pmatrix}
            0 & 1\\
            1 & 0
        \end{pmatrix}
        \begin{pmatrix}
            \cos{(Bt)}\\ 
            i \sin{(Bt)}
        \end{pmatrix}\\
        &= 2i \sin{Bt} \cos{Bt}.
    \end{split}
\]
The elements of the previous sum are given by
\[
    \begin{split}
        \langle \psi_t, \sigma_x^2 \psi_t \rangle = \begin{pmatrix}
            \cos{(Bt)} & -i \sin{(Bt)}
        \end{pmatrix}
        \begin{pmatrix}
            1 & 0\\
            0 & 1
        \end{pmatrix}
        \begin{pmatrix}
            \cos{(Bt)}\\ 
            i \sin{(Bt)}
        \end{pmatrix}
    \end{split} = 1
\]
\[
\begin{split}
     -4\langle \psi_t, \cos^2{Bt} \sin^2{Bt} \, \psi_t \rangle &= -4 \begin{pmatrix}
            \cos{(Bt)} & -i \sin{(Bt)}
        \end{pmatrix}
        \begin{pmatrix}
            \cos^3{(Bt)} \sin^2{Bt}\\ 
            i \sin^3{(Bt)} \cos^2{Bt}
        \end{pmatrix} = \\
        &= -4 \cos^2{Bt} \sin^2{Bt}
\end{split}
\]
\[
\begin{split}
     -4i\langle \psi_t, \sigma_x \cos{Bt} \sin{Bt} \, \psi_t \rangle &= -4 \begin{pmatrix}
            \cos{(Bt)} & -i \sin{(Bt)}
        \end{pmatrix}\cdot\\
        &\cdot\begin{pmatrix}
            0 & \cos{Bt}\sin{Bt}\\
            \cos{Bt}\sin{Bt} & 0
        \end{pmatrix}
        \begin{pmatrix}
            \cos{(Bt)}\\ 
            i \sin{(Bt)}
        \end{pmatrix} = 0
\end{split}
\]
Meaning that $(\Delta \sigma_x)_{\psi_t} = 1-4\cos^2{Bt}\sin^2{Bt} = \cos^2(2Bt)$. For $(\Delta \sigma_z)_{\psi_t}$ the procedure is the same, let
\[ 
    \begin{split}
        (\Delta \sigma_x)_{\psi_t} &= \langle \psi_t, (\sigma_z^2 + \cos^4{Bt} +\\
        &+ \sin^4{Bt} - 2\sigma_z \cos^2{Bt} +2 \sin^2{Bt} -2\cos^2{Bt}\sin^2{Bt})\, \psi_t \rangle
    \end{split}
\]
where we used the result from part c).
\[
    \begin{split}
        \langle \psi_t, \sigma_z^2 \psi_t \rangle = \begin{pmatrix}
            \cos{(Bt)} & -i \sin{(Bt)}
        \end{pmatrix}
        \begin{pmatrix}
            1 & 0\\
            0 & 1
        \end{pmatrix}
        \begin{pmatrix}
            \cos{(Bt)}\\ 
            i \sin{(Bt)}
        \end{pmatrix}
    \end{split} = 1
\]
\[
    \begin{split}
        \langle \psi_t, \cos^4{Bt}\, \psi_t \rangle &= \begin{pmatrix}
            \cos{(Bt)} & -i \sin{(Bt)}
        \end{pmatrix}
        \begin{pmatrix}
            \cos^5{(Bt)}\\ 
            i \sin{(Bt)} \cos^4{Bt}
        \end{pmatrix} =\\
        &= \cos^4{Bt}
    \end{split} 
\]
\[
    \begin{split}
        \langle \psi_t, \sin^4{Bt}\, \psi_t \rangle &= \begin{pmatrix}
            \cos{(Bt)} & -i \sin{(Bt)}
        \end{pmatrix}
        \begin{pmatrix}
            \cos{(Bt)}\sin^4{Bt}\\ 
            i \sin^5{(Bt)}
        \end{pmatrix}=\\
        &= \sin^4{Bt}
    \end{split}
\]
\[
    \begin{split}
       -2 \langle \psi_t, \sigma_z \cos^2{Bt}\, \psi_t \rangle &= -2 \begin{pmatrix}
            \cos{(Bt)} & -i \sin{(Bt)}
        \end{pmatrix}\cdot\\
        &\cdot\begin{pmatrix}
            \cos^2{Bt} & 0\\
            0 & \cos^2{Bt}
        \end{pmatrix}
        \begin{pmatrix}
            \cos{(Bt)}\\ 
            i \sin{(Bt)}
        \end{pmatrix}=\\
        &= -2\cos^4{Bt} +2 \sin^2{Bt}\cos^2{Bt}
    \end{split}
\]
\[
    \begin{split}
       2 \langle \psi_t, \sigma_z \sin^2{Bt}\, \psi_t \rangle &= 2 \begin{pmatrix}
            \cos{(Bt)} & -i \sin{(Bt)}
        \end{pmatrix}\cdot\\
        &\cdot\begin{pmatrix}
            \sin^2{Bt} & 0\\
            0 & \sin^2{Bt}
        \end{pmatrix}
        \begin{pmatrix}
            \cos{(Bt)}\\ 
            i \sin{(Bt)}
        \end{pmatrix}=\\
        &= - 2\sin^4{Bt} + 2 \sin^2{Bt}\cos^2{Bt}
    \end{split}
\]
\[
    \begin{split}
        -2 \langle \psi_t, \cos^2{Bt}\sin^2{Bt}\, \psi_t \rangle &= -2\begin{pmatrix}
            \cos{(Bt)} & -i \sin{(Bt)}
        \end{pmatrix}
        \begin{pmatrix}
            \cos^3{(Bt)}\sin^2{Bt}\\ 
            i \cos^2{Bt}\sin^3{(Bt)}
        \end{pmatrix}=\\
        &= -2 \cos^2{Bt}\sin^2{Bt}
    \end{split}
\]
So that $(\Delta \sigma_z)_{\psi_t} = 1-(\cos^2{Bt}-\sin^2{Bt})^2 = \sin^2(2Bt)$.
To find the minimum of $(\Delta \sigma_x)_{\psi_t} (\Delta \sigma_z)_{\psi_t}$ find the values for which its first derivative is zero:
\[
    \begin{split}
        \odv{}{t}(\Delta \sigma_x)_{\psi_t} (\Delta \sigma_z)_{\psi_t} &= \sin^2(2Bt) \cos^2{2Bt} =\\
        &= 4B\sin{2Bt}\cos{2Bt}\cos{4Bt} = 0
    \end{split}
\]
that is when $t_1 = n\pi$ or when $t_2 = n\pi/8B$ with $n \in \mathbb{Z}$. By looking at the graph of the function:
\begin{figure}[!ht]
    \centering
    \includegraphics[width = 0.7\textwidth]{graph.png}
\end{figure}\\
we see that $t_2$ is a maximum while the minimum is given by $t_1$.
\section{Solution:}
\subsection*{a)}
If we consider a function of the type $f(x)= (1-x^2)^{-p-1}$ it will be in $L^p$ since its integral is bounded but not in $L^q$. Vice versa if we take a function of the form $g(x)= (1-x^2)^{-q-1}$ it will be in $L^q$ but not in $L^p$.

\subsection*{b)}
Suppose f is supported in $B_R$, that is
\[
    \mbox{supp }f = \{ \mathbf{x} \in \mathbb{R} | f(\mathbf{x}) \neq 0 \} = B_R = \{ \mathbf{x}\in \mathbb{R} : |\mathbf{x}|<R \}
\]
meaning that if $f \in L^2(\mathbb{R}^d)$, then:
\[
    \int_{-R}^{R}\cdots \int_{-R}^{R} |f(\mathbf{x})|^2 d\mathbf{x} = \int_{-R}^{R}\cdots \int_{-R}^{R} |f(\mathbf{x})||f(\mathbf{x})| d\mathbf{x}.
\]
Which is bounded from the top as
\[
    \int_{-R}^{R}\cdots \int_{-R}^{R} |f(\mathbf{x})||f(\mathbf{x})| d\mathbf{x} < \int_{-R}^{R}\cdots \int_{-R}^{R} |f(\mathbf{x})|C d\mathbf{x} <\infty,
\]
since it $f\in L^2(\mathbb{R}^d)$
\[
    \int_{-R}^{R}\cdots \int_{-R}^{R} |f(\mathbf{x})| d\mathbf{x} < \int_{-R}^{R}\cdots \int_{-R}^{R}C d\mathbf{x} <\infty
\]
\subsection*{c)}
The fact that a function $f \in L_1 \cap L^\infty$ means that at the same time
\[
    \int_{\mathbb{R}^d} |f(\mathbf{x})| d\mathbf{x}< \infty,
\]
and
\[
    \sup_{\mathbf{x} \in \mathbb{R}^d} |f(\mathbf{x})| < \infty .
\]
Since the supremum of the function is bounded, meaning that there exist a $C<\infty$ such that $\sup_{ \mathbf{x} \in \mathbb{R}^d }|f(\mathbf{x})|<C, \; \forall \mathbf{x}$. This means that the integral
\[
    \int_{\mathbb{R}^d} |f(\mathbf{x})|^p d\mathbf{x}
\]
is also bounded, since $|f(\mathbf{x})|< \sup_{ \mathbf{x} \in \mathbb{R}^d }|f(\mathbf{x})| < C$ and therefore
\[
     \int_{\mathbb{R}^d} |f(\mathbf{x})|^p d\mathbf{x} <  \int_{\mathbb{R}^d} |f(\mathbf{x})| C^{p-1} d\mathbf{x} < \infty
\]
\subsection*{d)}
Let $f \in \mathcal{S}$ then $\sup |x^\alpha {\partial^\beta_x f}|<\infty$ that is $||x^\alpha {\partial^\beta_x f}||_\infty$. Now let us consider:
\[
    \begin{split}
        \int_{\mathbb{R^d}} |f(x)|^p dx &= \int_{\mathbb{R^d}} |f(x) \frac{x^{p+1}}{x^{p+1}}|^p dx = \\
        &= \int_{\mathbb{R^d}} |f(x) x^{p+1}|^p \frac{1}{x}dx \leq || f(x) x^{p+1} ||_{\infty} \int_{\mathbb{R}^d} \frac{1}{x}dx < \infty
    \end{split}
\]
since by hypothesis $\sup |x^\alpha {\partial^\beta_x f}|<\infty$ and the integral converges.
\subsection*{e)}
Let $f \in \mathcal{S}$ and consider a polynomial $p: \mathbb{R}^d \to \mathbb{C}$ then for $pf$ to be in $\mathcal{S}$, it means that
\[
    \sup|x^\alpha \partial_x^\beta (pf)|<\infty
\]
which using the chain rule becomes
\[
    \sup|x^\alpha \partial_x^\beta (pf)| = \sup|x^\alpha p \partial_x^\beta f| +\sup|x^\alpha f \partial_x^\beta p|<\infty
\]
since the derivative of a polynomial is still a polynomial and by hypothesis $\sup|x^\alpha \partial_x^\beta (f)|<\infty$.
\section{Solution:}
$(\Rightarrow)$ Let us assume that $d_S (f_j -f) \rightarrow 0$, then:
\[
    \forall \epsilon >0, \exists\, N \in \mathbb N : \forall j>N \, | \, d_S(f_j -f) < \epsilon
\]
meaning that, by definition,
\[
    \sum_{n=0}^\infty 2^{-n} \sup_{\substack{|\alpha|+|\beta| = n \\ \alpha, \beta \in \mathbb N_0^d}} \frac{|| f_j - f||_{\alpha\beta}}{1+||f_j - f||_{\alpha, \beta}} < \epsilon.
\]
From the fact that the $\sup_{x \in \Omega} g(x) \geq g(x)$ and since $\sum_{n=0}^\infty 2^{-2} > 1/2$ then we can say that,
\[
    \frac{1}{2} \frac{|| f_j - f||_{\alpha\beta}}{1+||f_j - f||_{\alpha, \beta}} < \epsilon.
\]
Acting on the inequality,
\[
    \frac{1}{2}|| f_j - f||_{\alpha\beta} < (1+||f_j - f||_{\alpha, \beta}) \epsilon.
\]
\[
    || f_j - f||_{\alpha\beta} < \frac{\epsilon}{\frac{1}{2}-\epsilon} = \bar{\epsilon}.
\]
Proving the first direction of the statement.\\
$(\Leftarrow)$
In this case we know that there for all $\epsilon>0$ there exists an $N>0$ such that for $n>N$, $||f - f_n||_{\alpha,\beta}<\epsilon$. From this we know that the norm is define to be non-negative meaning that
\[
    \frac{||f - f_n||_{\alpha,\beta}}{1 + ||f - f_n||_{\alpha,\beta}} < ||f - f_n||_{\alpha,\beta}.
\]
In addition to that,

\section{Solution:}
Let us consider the function $g_\lambda (\mathbf{x}) = e^{-\lambda \frac{|\mathbf{x}|^2}{2}}$ such that $g_\lambda (\mathbf{x}) \in L^1(\mathbb{R}^d)$. To compute the Fourier transform of this function $\hat{g}_\lambda (\mathbf{k})$ we need to compute:
\[
    \hat{g}_\lambda (\mathbf{k}) = \frac{1}{(2 \pi)^{d/2}} \int_\mathbb{R}{ e^{-i\mathbf{k \cdot x}} g_\lambda (\mathbf{x})d\mathbf{x}}.
\]
Then by focusing only on the integral we will have that
\[
    \begin{split}
        I &= \int_{\mathbb R^d} e^{-i \mathbf{k \cdot x}} e^{-\lambda \frac{|\mathbf{x}|^2}{2}} d\mathbf{x} = \int_{\mathbb R^d} e^{-i \mathbf{k \cdot x} - \lambda \frac{|\mathbf{x}|^2}{2}} d\mathbf{x} =\\
	   &= \int_{\mathbb R^d} e^{-(\sqrt{\frac{\lambda}{2}} \mathbf{x} + \frac{1}{\sqrt{2\lambda}}\mathbf{k})^2} e^{-\frac{|\mathbf{k}|^2}{2\lambda}} d\mathbf{x},
    \end{split}
\]
in which we completed the square in the exponent. Let us now define $\mathbf{u} = \sqrt{\frac{\lambda}{2}} \mathbf{x} + \frac{1}{\sqrt{2\lambda}}\mathbf{k}$ such that $u_i = \sqrt{\frac{\lambda}{2}} x_i + \frac{1}{\sqrt{2\lambda}}k_i$, then $du_i = \sqrt{\frac{\lambda}{2}} dx_i$.

Since $|\mathbf{u}|^2 = \sum^d_{i=1} u_i^2$, if we assume that each component is independent from the others, then it is possible to separate the integral in the product of $d$ integrals
\[
    \prod^d_{i=0} \int_{-\infty}^{\infty} \sqrt{\frac{2}{\lambda}} e^{-u_i^2}du_i = \left( \frac{2}{\lambda} \right)^{d/2} \int_{\mathbb R^d} e^{-|\mathbf{u}|^2} d\mathbf{u}.
\]
Moreover, we notice that the integrals of each component are of the form of Gauss's integral, we conclude that the I must be
\[
    I = \left( \frac{2 \pi}{\lambda} \right)^{d/2} e^{-\frac{|\mathbf{k}|^2}{2\lambda}}.
\]
Meaning that the Fourier transform of $g_\lambda (\mathbf{x})$ must be:
\[
    \hat{g}_\lambda (\mathbf{k}) = \lambda^{-d/2} e^{-\frac{|\mathbf{k}|^2}{2\lambda}}.
\]

\end{document}
